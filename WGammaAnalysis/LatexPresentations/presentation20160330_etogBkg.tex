\documentclass{beamer}
\usepackage[latin1]{inputenc}
\usepackage{color}
\usepackage{multirow}
%\usepackage[pdftex]{graphicx}
%\usepackage{sidecap}
%\usepackage{siunitx}
%\usepackage{epstopdf}
\usetheme{default}
\title{WGamma. $e\rightarrow\gamma$ Background Estimation for the Electron Channel}
\author{Ekaterina Avdeeva}
\institute{University of Nebraska - Lincoln}
\date{March 30th, 2016}



\begin{document}

\begin{frame}
\titlepage
\end{frame}

\begin{frame}\frametitle{Wg Signature and Selection}

\end{frame}

\begin{frame}\frametitle{Main Source of the e$\rightarrow\gamma$ Background}

\end{frame}

\begin{frame}\frametitle{Simple Counting with Inverted PSV Cut. Method}

\end{frame}

\begin{frame}\frametitle{Simple Counting with Inverted PSV Cut. Concern}

\end{frame}

\begin{frame}\frametitle{Simple Counting with Nominal PSV Cut. Plots}

\end{frame}

\begin{frame}\frametitle{From Simple Counting to Fit of Data}

\end{frame}

\begin{frame}\frametitle{Fit Model}
\scriptsize
first: BW x CB + Exponential\\
About signal model: BW x CB -> RooNDKeysPdf x Gaussian\\
About background model: Exponential -> RooCMSShapePdf\\
finally, fit model:\\
RooNDKeysPdf x Gaussian + RooCMSShapePdf\\
\end{frame}

%\begin{frame}\frametitle{WGamma Data MUON Barrel}
%  \includegraphics[width=0.38\textwidth, keepaspectratio=true]{../WGammaOutput/MUON_WGamma/Plots/PrepareYields/c_BkgSubtrDATAvsSIGMC_c_MUON_WGamma__blindCOMBINED__Barrel__phoEt.png} \includegraphics[width=0.38\textwidth, keepaspectratio=true]{../WGammaOutput/MUON_WGamma/Plots/PrepareYields/c_FakeDDvsMC_c_MUON_WGamma__blindCOMBINED__Barrel__phoEt.png}\\
%  \includegraphics[width=0.38\textwidth, keepaspectratio=true]{../../WGammaAnalysisAux48_AddMatrixMethod/WGammaOutput/MUON_WGamma/Plots/PrepareYields/c_BkgSubtrDATAvsSIGMC_c_MUON_WGamma__blindCOMBINED__Barrel__phoEt.png} \includegraphics[width=0.38\textwidth, keepaspectratio=true]{../../WGammaAnalysisAux48_AddMatrixMethod/WGammaOutput/MUON_WGamma/Plots/PrepareYields/c_FakeDDvsMC_c_MUON_WGamma__blindCOMBINED__Barrel__phoEt.png}\\
%  \scriptsize Top: fits, bottom: matrix method.
%\end{frame}



\end{document}
